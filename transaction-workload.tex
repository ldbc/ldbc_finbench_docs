\chapter{Transaction Workload}
\label{sec:transaction-workload}

This workload consists of a set of relatively simple read queries, write queries
and read-write operations, that touch a significant amount of data. These
queries and operations are usually considered as online data processing and
analysis in online financial systems. The LDBC FinBench transaction workload
consists of four query types:
\begin{itemize}
    \item Complex read queries. See \autoref{sec:complex-read-queries}. This
          section contains many basic read queries that are typical in financial
          scenarios.
    \item Simple read queries. See \autoref{sec:simple-read-queries}. This
          section contains many basic read queries that are typical in financial
          scenarios.
    \item Write queries. See \autoref{sec:write-queries}. This section contains
          many basic write queries that are typical in financial scenarios.
    \item Read-write queries. See \autoref{sec:rw-queries}. This section
          contains many read-write operations composed of basic reads and writes.
\end{itemize}

{\flushleft \textbf{About the truncation:}} Among the queries, a mechanism is
introduced that helps handling the discordance between the tight latency requirements
and power-law distribution of data in system, where the degree of hub vertex may
reach to million and even billion scale, especially when traversing on graph.
The mechanism is to do truncation on the edges when traversing out from the
current vertex, which is in compliance with the discordance in fact. However, we
would like to introduce this mechanism because the high degree of hub vertex is
a common feature not only in financial scenarios but also in other scenarios,
which is an inevitable challenge that systems face. The mechanism is also called
\textbf{PER\_NODE\_LIMIT} somewhere. To solve the problem, systems can either
improve the performance to satisfy the computation or just cut down the
complexity to meet the latency requirements, where truncation is a common and
useful mechanism. To maintain the consistency of the results, a sort order has
to be specified when truncating. Since in financial graphs, users prefer newer
data in business. It is reasonable that attribute, \textbf{timestamp}, in the
edges is used as the sort order in truncation. In the following queries, some
parameters are added to describe the behavior of truncation cutting down the
complexity including the \textbf{TRUNCATION\_LIMIT} and \textbf{TRUNCATION\_ORDER}.
\textbf{TRUNCATION\_ORDER} can be \textbf{TIME\_ASCENDING, TIME\_DESCENDING,
AMOUNT\_ASCENDING, AMOUNT\_DESCENDING}. At most time, \textbf{TRUNCATION\_ORDER}
is set to \textbf{TIME\_DESCENDING} by default.

{\flushleft \textbf{About the annotations:}} To express the patterns better, the
pattern diagrams are drawn from the perspective of data rather than the matching
pattern in graph. Here are some annotations to each query card in this section.
\begin{itemize}
      \item Each row in the result cell represents an attribute to be returned.
      \item The second column means the data type of returned attribute.
      If the type is surrounded by \type{\{\}}, it means that the result is a
      set, \eg \type{\{String\}} means a string set is returned.
      \item For each row in the result cell, the third column annotates the
      category of type of result attribute returned, including \texttt{R} short
      for Raw, \texttt{A} short for Aggregated, \texttt{C} short for Calculated,
      \texttt{S} short for Structural. Among them, structural type means types
      such as \type{Path} while raw type means basic types in contrast.
\end{itemize}


\section{Complex Read Queries}
\label{sec:complex-read-queries}

\input{query-cards/transaction-complex-read-01}
\input{query-cards/transaction-complex-read-02}
% \input{query-cards/transaction-complex-read-03}
\input{query-cards/transaction-complex-read-04}
% \input{query-cards/transaction-complex-read-05}
% \input{query-cards/transaction-complex-read-06}
\input{query-cards/transaction-complex-read-07}
\input{query-cards/transaction-complex-read-08}
% \input{query-cards/transaction-complex-read-09}
\input{query-cards/transaction-complex-read-10}
\input{query-cards/transaction-complex-read-11}
% \input{query-cards/transaction-complex-read-12}
\input{query-cards/transaction-complex-read-13}
\input{query-cards/transaction-complex-read-14}
\input{query-cards/transaction-complex-read-15}
% \input{query-cards/transaction-complex-read-16}
\input{query-cards/transaction-complex-read-17}
% \input{query-cards/transaction-complex-read-18}
\input{query-cards/transaction-complex-read-19}
% \input{query-cards/transaction-complex-read-20}
% \input{query-cards/transaction-complex-read-21}
% \input{query-cards/transaction-complex-read-22}
% \input{query-cards/transaction-complex-read-23}
% \input{query-cards/transaction-complex-read-24}

%%%%%%%%%%%%%%%%%%%%%%%%%%%%%%%%%%%%%%%%%%%%%%%%%%%%%%%%%%%%%%%%%%%%%%%%%%%%%%
%%%%%%%%%%%%%%%%%%%%%%%%%%%%%%%%%%%%%%%%%%%%%%%%%%%%%%%%%%%%%%%%%%%%%%%%%%%%%%
%%%%%%%%%%%%%%%%%%%%%%%%%%%%%%%%%%%%%%%%%%%%%%%%%%%%%%%%%%%%%%%%%%%%%%%%%%%%%%

\section{Simple Read Queries}
\label{sec:simple-read-queries}

\input{query-cards/transaction-simple-read-01}
\input{query-cards/transaction-simple-read-02}
\input{query-cards/transaction-simple-read-03}
\input{query-cards/transaction-simple-read-04}
\input{query-cards/transaction-simple-read-05}
\input{query-cards/transaction-simple-read-06}
% \input{query-cards/transaction-simple-read-07}
% \input{query-cards/transaction-simple-read-08}

%%%%%%%%%%%%%%%%%%%%%%%%%%%%%%%%%%%%%%%%%%%%%%%%%%%%%%%%%%%%%%%%%%%%%%%%%%%%%%
%%%%%%%%%%%%%%%%%%%%%%%%%%%%%%%%%%%%%%%%%%%%%%%%%%%%%%%%%%%%%%%%%%%%%%%%%%%%%%
%%%%%%%%%%%%%%%%%%%%%%%%%%%%%%%%%%%%%%%%%%%%%%%%%%%%%%%%%%%%%%%%%%%%%%%%%%%%%%

\section{Write Queries}
\label{sec:write-queries}

\input{query-cards/transaction-write-01}
\input{query-cards/transaction-write-02}
\input{query-cards/transaction-write-03}
\input{query-cards/transaction-write-04}
\input{query-cards/transaction-write-05}
\input{query-cards/transaction-write-06}
\input{query-cards/transaction-write-07}
\input{query-cards/transaction-write-08}
\input{query-cards/transaction-write-09}
\input{query-cards/transaction-write-10}
\input{query-cards/transaction-write-11}
\input{query-cards/transaction-write-12}
\input{query-cards/transaction-write-13}
\input{query-cards/transaction-write-14}

%%%%%%%%%%%%%%%%%%%%%%%%%%%%%%%%%%%%%%%%%%%%%%%%%%%%%%%%%%%%%%%%%%%%%%%%%%%%%%
%%%%%%%%%%%%%%%%%%%%%%%%%%%%%%%%%%%%%%%%%%%%%%%%%%%%%%%%%%%%%%%%%%%%%%%%%%%%%%
%%%%%%%%%%%%%%%%%%%%%%%%%%%%%%%%%%%%%%%%%%%%%%%%%%%%%%%%%%%%%%%%%%%%%%%%%%%%%%

\section{Read-Write Queries}
\label{sec:rw-queries}

In financial scenario, risk control is a kind of hot and significant application.
Such applications usually detect a specific pattern in form of linked data before
new records like transfers are written to systems. Read-write query, which can also
be seen as transaction-wrapped strategies, fits these applications very well since
users do not need to worry about translating the patterns to prevent malicious records.
Read-write query is composed of read queries and write queries in the previous sections.
At most cases, whether to commit write query depends on the detection result of the
read queries. In the initial version, just 3 read-write queries are presented.

\input{query-cards/transaction-read-write-01}
\input{query-cards/transaction-read-write-02}
\input{query-cards/transaction-read-write-03}

%%%%%%%%%%%%%%%%%%%%%%%%%%%%%%%%%%%%%%%%%%%%%%%%%%%%%%%%%%%%%%%%%%%%%%%%%%%%%%
%%%%%%%%%%%%%%%%%%%%%%%%%%%%%%%%%%%%%%%%%%%%%%%%%%%%%%%%%%%%%%%%%%%%%%%%%%%%%%
%%%%%%%%%%%%%%%%%%%%%%%%%%%%%%%%%%%%%%%%%%%%%%%%%%%%%%%%%%%%%%%%%%%%%%%%%%%%%%
