\chapter{Workloads}
\label{sec:workloads}

\section{Query Annotations}

This section describes how to read the query cards in the following sections.
\subsection{Query Description Format}
\label{subsec:query-description-format}
 [TODO. This section will be filled further after draft is approved.]

%%%%%%%%%%%%%%%%%%%%%%%%%%%%%%%%%%%%%%%%%%%%%%%%%%%%%%%%%%%%%%%%%%%%%%%%%%%%%%
%%%%%%%%%%%%%%%%%%%%%%%%%%%%%%%%%%%%%%%%%%%%%%%%%%%%%%%%%%%%%%%%%%%%%%%%%%%%%%
%%%%%%%%%%%%%%%%%%%%%%%%%%%%%%%%%%%%%%%%%%%%%%%%%%%%%%%%%%%%%%%%%%%%%%%%%%%%%%

\subsection{Substitution Parameters}
\label{subsec:substitution-parameters}
 [TODO. This section will be filled further after draft is approved.]

%%%%%%%%%%%%%%%%%%%%%%%%%%%%%%%%%%%%%%%%%%%%%%%%%%%%%%%%%%%%%%%%%%%%%%%%%%%%%%
%%%%%%%%%%%%%%%%%%%%%%%%%%%%%%%%%%%%%%%%%%%%%%%%%%%%%%%%%%%%%%%%%%%%%%%%%%%%%%
%%%%%%%%%%%%%%%%%%%%%%%%%%%%%%%%%%%%%%%%%%%%%%%%%%%%%%%%%%%%%%%%%%%%%%%%%%%%%%

\subsection{Returned Values}
\label{subsec:returned-values}

Return values are subject to the following rules:
\begin{itemize}
    \item Path type. The Path type is a sequence of vertices and edges. The
          Path type is returned as a sequence of vertex and edge identifiers
          ignoring the multiple edges between the same src and dst vertex.
\end{itemize}

%%%%%%%%%%%%%%%%%%%%%%%%%%%%%%%%%%%%%%%%%%%%%%%%%%%%%%%%%%%%%%%%%%%%%%%%%%%%%%
%%%%%%%%%%%%%%%%%%%%%%%%%%%%%%%%%%%%%%%%%%%%%%%%%%%%%%%%%%%%%%%%%%%%%%%%%%%%%%
%%%%%%%%%%%%%%%%%%%%%%%%%%%%%%%%%%%%%%%%%%%%%%%%%%%%%%%%%%%%%%%%%%%%%%%%%%%%%%

\subsection{Other Annotations}
\label{subsec:other-annotations}

To express the patterns better, the pattern diagrams are drawn from the perspective
of data rather than the matching pattern in graph. Here are some annotations to each
query card in this section.
\begin{itemize}
    \item Each row in the result cell represents an attribute to be returned.
    \item The second column means the data type of returned attribute.
          If the type is surrounded by \type{\{\}}, it means that the result is a
          set, \eg \type{\{String\}} means a string set is returned.
    \item For each row in the result cell, the third column annotates the
          category of type of result attribute returned, including \texttt{R} short
          for Raw, \texttt{A} short for Aggregated, \texttt{C} short for Calculated,
          \texttt{S} short for Structural. Among them, structural type means types
          such as \type{Path} while raw type means basic types in contrast.
\end{itemize}

%%%%%%%%%%%%%%%%%%%%%%%%%%%%%%%%%%%%%%%%%%%%%%%%%%%%%%%%%%%%%%%%%%%%%%%%%%%%%%
%%%%%%%%%%%%%%%%%%%%%%%%%%%%%%%%%%%%%%%%%%%%%%%%%%%%%%%%%%%%%%%%%%%%%%%%%%%%%%
%%%%%%%%%%%%%%%%%%%%%%%%%%%%%%%%%%%%%%%%%%%%%%%%%%%%%%%%%%%%%%%%%%%%%%%%%%%%%%

\section{Load Pattern of Transaction Workload}
 [TODO. This section will be designed further after draft is approved. The design concepts are listed in Section 1.2.2]

%%%%%%%%%%%%%%%%%%%%%%%%%%%%%%%%%%%%%%%%%%%%%%%%%%%%%%%%%%%%%%%%%%%%%%%%%%%%%%
%%%%%%%%%%%%%%%%%%%%%%%%%%%%%%%%%%%%%%%%%%%%%%%%%%%%%%%%%%%%%%%%%%%%%%%%%%%%%%
%%%%%%%%%%%%%%%%%%%%%%%%%%%%%%%%%%%%%%%%%%%%%%%%%%%%%%%%%%%%%%%%%%%%%%%%%%%%%%

\section{Truncation on Hub Vertices}
[TODO: fix issue 71]

Among the queries, a mechanism is introduced that helps handling the discordance
between the tight latency requirements
and power-law distribution of data in system, where the degree of hub vertex may
reach to million and even billion scale, especially when traversing on graph.
The mechanism is to do truncation on the edges when traversing out from the
current vertex, which is in compliance with the discordance in fact. However, we
would like to introduce this mechanism because the high degree of hub vertex is
a common feature not only in financial scenarios but also in other scenarios,
which is an inevitable challenge that systems face. The mechanism is also called
\textbf{PER\_NODE\_LIMIT}. To solve the problem, systems can either
improve the performance to satisfy the computation or just reduce the
complexity to meet the latency requirements, where truncation is a common and
useful mechanism. To maintain the consistency of the results, a sort order has
to be specified when truncating. Since in financial graphs, users prefer newer
data in business. It is reasonable that attribute, \textbf{timestamp}, in the
edges is used as the sort order in truncation. In the following queries, some
parameters are added to describe the behavior of truncation reducing the
complexity including the \textbf{TRUNCATION\_LIMIT} and \textbf{TRUNCATION\_ORDER}.
\textbf{TRUNCATION\_ORDER} can be \textbf{TIME\_ASCENDING, TIME\_DESCENDING,
AMOUNT\_ASCENDING, AMOUNT\_DESCENDING}. At most time, \textbf{TRUNCATION\_ORDER}
is set to \textbf{TIME\_DESCENDING} by default.

%%%%%%%%%%%%%%%%%%%%%%%%%%%%%%%%%%%%%%%%%%%%%%%%%%%%%%%%%%%%%%%%%%%%%%%%%%%%%%
%%%%%%%%%%%%%%%%%%%%%%%%%%%%%%%%%%%%%%%%%%%%%%%%%%%%%%%%%%%%%%%%%%%%%%%%%%%%%%
%%%%%%%%%%%%%%%%%%%%%%%%%%%%%%%%%%%%%%%%%%%%%%%%%%%%%%%%%%%%%%%%%%%%%%%%%%%%%%

\section{Read Write Query}

In financial scenario, risk control is a kind of hot and significant application.
Such applications usually detect a specific pattern in form of linked data before
new records like transfers are written to systems. Read-write query, which can also
be seen as transaction-wrapped strategies, fits these applications very well since
users do not need to worry about translating the patterns to prevent malicious records.
Read-write query is composed of read queries and write queries in the previous sections.
At most cases, whether to commit write query depends on the detection result of the
read queries. In the initial version, just 3 read-write queries are presented.