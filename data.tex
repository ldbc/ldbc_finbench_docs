\chapter{Data Definition}
\label{sec:data-definition}

This chapter describes the dataset used by \ldbcfinbench, including the data
schema design and the data generation process. Generally, we design
\ldbcfinbench balancing the reality and abstraction. There are some annotations
about the compromises in data design,
\begin{itemize}
    \item Although multiple persons/companies may own the same account in
          reality, in the schema, an account is owned by only a single person
          or company for simplicity.
    \item Although rejected transactions may be recorded to support future loan
          decisions, only approved transactions/transfers are recorded in the
          benchmark dataset.
    \item Considering the number of daily active users (DAU) of financial systems
          in reality, there will be
          many signIn edges between medium and account vertices. However, we do
          not generate so many signIn edges aligning to reality with a limit
          in the simulation of data generation process since systems usually
          circumvent the problem by adding a medium attribute to edges like
          transfer and withdraw to record the medium users used.
\end{itemize}

\section{Data Types}

\autoref{table:types} describes the different data types used in the benchmark.
Compared with \ldbcsnb, there is a new compound type, \textbf{Path}, which is
widely applied in financial scenarios reflecting traces, \eg fund transfer
traces.

\begin{table}[h]
    \centering
    \begin{tabular}{|>{\typeCell}p{\attributeColumnWidth}|p{\largeDescriptionColumnWidth}|}
        \hline
        \tableHeaderFirst{Type} & \tableHeader{Description}                      \\
        \hline
        ID                      & Integer type with 64-bit precision. All IDs
        within a single entity type (\eg Person) are unique, but different
        entity types (\eg a Person and an Account) might have the same ID.       \\
        \hline
        32-bit Integer          & Integer type with 32-bit precision             \\
        \hline
        64-bit Integer          & Integer type with 64-bit precision             \\
        \hline
        32-bit Float            & Floating type with 32-bit precision            \\
        \hline
        64-bit Float            & Floating type with 64-bit precision            \\
        \hline
        String                  & Variable length text of size 40 Unicode
        characters                                                               \\
        \hline
        Long String             & Variable length text of size 256 Unicode
        characters                                                               \\
        \hline
        Text                    & Variable length text of size 2000 Unicode
        characters                                                               \\
        \hline
        Date                    & Date with a precision of a day, encoded as a
        string with the following format: \textit{yyyy-mm-dd}, where
        \textit{yyyy} is a four-digit integer representing the year, the year,
        \textit{mm} is a two-digit integer representing the month and
        \textit{dd} is a two-digit integer representing the day.                 \\
        \hline
        DateTime                & Date with a precision of milliseconds, encoded
        as a string with the following format:
        \textit{yyyy-mm-ddTHH:MM:ss.sss+0000}, where \textit{yyyy} is a
        four-digit integer representing the year, the year, \textit{mm} is a
        two-digit integer representing the month and \textit{dd} is a two-digit
        integer representing the day, \textit{HH} is a two-digit integer
        representing the hour, \textit{MM} is a two digit integer representing
        the minute and \textit{ss.sss} is a five digit fixed point real number
        representing the seconds up to millisecond precision. Finally, the
        \textit{+0000} of the end represents the timezone, which in this case is
        always GMT.                                                              \\
        \hline
        Boolean                 & A logical type taking the value of either True
        of False.                                                                \\
        \hline
        Enum                    & Enumeration type                               \\
        \hline
        Path                    & A compound type representing a trace which is
        expressed in an ordered sequence of vertices' IDs in the trace. For
        example, [1,3,4,8] expresses a trace 1->3->4->8.                         \\
        \hline
    \end{tabular}
    \caption{Description of the data types.}
    \label{table:types}
\end{table}

\subsection{Enumerations}
{\flushleft \textbf{TRUNCATION\_ORDER:}} The enumeration describes the sort
order before truncation. \textbf{TIMESTAMP\_ASCENDING} means truncation on
ascending order of timestamp.


\section{Data Schema}

\autoref{figure:schema} shows the data schema in UML. The schema defines the
structure of the data used in the benchmark in terms of entities and their
relations. The data represents a snapshot of the activity in several financial
scenarios during a period of time. The schema specifies different entities,
their attributes, and their relations. All of them are described in the
following sections.

\begin{figure}[htbp]
    \centering
    \includegraphics[width=\linewidth]{figures/data-schema}
    \caption{The \ldbcfinbench data schema}
    \label{figure:schema}
\end{figure}

\subsection{Entities}

{\flushleft \textbf{Person:}} A person of the real world. \autoref{table:person}
shows the attributes.
\begin{table}[H]
    \begin{tabular}{|>{\varNameCell}p{\attributeColumnWidth}|>{\typeCell}p{\typeColumnWidth}|p{\descriptionColumnWidth}|}
        \hline
        \tableHeaderFirst{Attribute} & \tableHeader{Type} &
        \tableHeader{Description}                                                                             \\
        \hline
        id                           & ID                 & The identifier of the person.                     \\
        \hline
        name                         & String             & The name of the person.                           \\
        \hline
        isBlocked                    & Boolean            & If the person is blocked or concerned in systems. \\
        \hline
        createTime                   & DateTime           & The time when the person created.                 \\
        \hline
        gender                       & String             & Gender of the person                              \\
        \hline
        birthday                     & Date               & Birthday of the person                            \\
        \hline
        country                      & String             & Country of the person                             \\
        \hline
        city                         & String             & City of the person                                \\
        \hline
    \end{tabular}
    \caption{Attributes of Person entity.}
    \label{table:person}
\end{table}

{\flushleft \textbf{Company:}} A company of the real world, which persons or other companies invest in. \autoref{table:company} shows the attributes.
\begin{table}[H]
    \begin{tabular}{|>{\varNameCell}p{\attributeColumnWidth}|>{\typeCell}p{\typeColumnWidth}|p{\descriptionColumnWidth}|}
        \hline
        \tableHeaderFirst{Attribute} & \tableHeader{Type} &
        \tableHeader{Description}                                                                              \\
        \hline
        id                           & ID                 & The identifier of the company.                     \\
        \hline
        name                         & String             & The name of the company.                           \\
        \hline
        isBlocked                    & Boolean            & If the company is blocked or concerned in systems. \\
        \hline
        createTime                   & DateTime           & The time when the company created.                 \\
        \hline
        country                      & String             & Country of the company                             \\
        \hline
        city                         & String             & City of the company                                \\
        \hline
        business                     & String             & The main business of the company                   \\
        \hline
        description                  & Long String        & The description of the company                     \\
        \hline
        url                          & String             & The url of the company's official site             \\
        \hline
    \end{tabular}
    \caption{Attributes of Company entity.}
    \label{table:company}
\end{table}

{\flushleft \textbf{Account:}} An account in real-world financial systems, which
is registered and owned by persons and companies. It includes many types such as
personalDeposit, personalCredit, etc. It can deal with other accounts.
\autoref{table:account} shows the attributes.
\begin{table}[H]
    \begin{tabular}{|>{\varNameCell}p{\attributeColumnWidth}|>{\typeCell}p{\typeColumnWidth}|p{\descriptionColumnWidth}|}
        \hline
        \tableHeaderFirst{Attribute} & \tableHeader{Type} &
        \tableHeader{Description}                                                                              \\
        \hline
        id                           & ID                 & The identifier of the account.                     \\
        \hline
        createTime                   & DateTime           & The time when the account created.                 \\
        \hline
        isBlocked                    & Boolean            & If the account is blocked or concerned in systems. \\
        \hline
        type                         & String             & The type of the account.                           \\
        \hline
        nickname                     & String             & The nickname of the account.                       \\
        \hline
        phoneNumber                  & String             & The phone number of the account.                   \\
        \hline
        email                        & String             & The email of the account.                          \\
        \hline
        freqLoginType                & String             & The frequent login type of the account.            \\
        \hline
        lastLoginTime                & DateTime           & The last login time of the account.                \\
        \hline
        accountLevel                 & String             & The level of the account.                          \\
        \hline
    \end{tabular}
    \caption{Attributes of Account entity.}
    \label{table:account}
\end{table}

{\flushleft \textbf{Loan:}} A loan for persons and company to apply in real
world. \autoref{table:loan} shows the attributes.
\begin{table}[H]
    \begin{tabular}{|>{\varNameCell}p{\attributeColumnWidth}|>{\typeCell}p{\typeColumnWidth}|p{\descriptionColumnWidth}|}
        \hline
        \tableHeaderFirst{Attribute} & \tableHeader{Type} &
        \tableHeader{Description}                                                        \\
        \hline
        id                           & ID                 & The identifier of the loan.  \\
        \hline
        loanAmount                   & 64-bit Float       & The amount of a loan.        \\
        \hline
        balance                      & 64-bit Float       & The balance of a loan.       \\
        \hline
        usage                        & String             & The usage of a loan.         \\
        \hline
        interestRate                 & 32-bit Float       & The interest rate of a loan. \\
        \hline
    \end{tabular}
    \caption{Attributes of Loan entity.}
    \label{table:loan}
\end{table}

{\flushleft \textbf{Medium:}} An abstract standing for things that users use to
sign in account in the real world, such as IP address, MAC address, phone numbers.
\autoref{table:medium} shows the attributes.
\begin{table}[H]
    \begin{tabular}{|>{\varNameCell}p{\attributeColumnWidth}|>{\typeCell}p{\typeColumnWidth}|p{\descriptionColumnWidth}|}
        \hline
        \tableHeaderFirst{Attribute} & \tableHeader{Type} & \tableHeader{Description}                         \\
        \hline
        id                           & ID                 & The identifier of the medium.                     \\
        \hline
        type                         & String             & The medium type, \eg POS, IP.                     \\
        \hline
        createTime                   & DateTime           & The time when the medium created.                 \\
        \hline
        isBlocked                    & Boolean            & If the medium is blocked or concerned in systems. \\
        \hline
        lastLoginTime                & DateTime           & The last login time of the medium.                \\
        \hline
        riskLevel                    & String             & The risk level of the medium.                     \\
        \hline
    \end{tabular}
    \caption{Attributes of Medium entity.}
    \label{table:medium}
\end{table}

\subsection{Relations}
Relations connect entities of different types showed in
\autoref{table:relations}. Except that own has no attributes, the
attributes of other relations are showed in the following tables. Note that the
Cardinality means the cardinal relationship from the tail to head of the edge
type and the Multiplicity means how many edges exist from the same tail to the
same head. For example, the 1 : N cardinality of own means an account can
only be owned by a person or a company.

\begin{longtable}{|>{\centering\varNameCell}p{1.5cm}|>{\typeCell}p{1.5cm}|>{\centering\cardinalCell}p{2cm}|>{\typeCell}p{1.5cm}|>{\centering\edgeDirectionCell}p{2cm}|p{5.5cm}|}
    \hline
    \tableHeaderFirst{Name} & \tableHeader{Tail}       & \tableHeader{Cardinality} & \tableHeader{Head}       & \tableHeader{Multiplicity} & \tableHeader{Description}                                            \\
    \hline
    signIn                  & Medium                   & N:N                       & Account                  & N                          & An account signed in with a media.                                   \\
    \hline
    own                     & Person/ \newline Company & 1:N                       & Account                  & 1                          & An account owned by a person or a company.                           \\
    \hline
    transfer                & Account                  & N:N                       & Account                  & N                          & Fund transferred between two accounts.                               \\
    \hline
    withdraw                & Account                  & N:N                       & Account                  & N                          & Fund transferred from an account to another account of type card.    \\
    \hline
    apply                   & Person/ \newline Company & 1:N                       & Loan                     & 1                          & A person or a company applies a Loan.                                \\
    \hline
    deposit                 & Loan                     & N:N                       & Account                  & N                          & Loan fund deposited to an account.                                   \\
    \hline
    repay                   & Account                  & N:N                       & Loan                     & N                          & Loan repaid from an account.                                         \\
    \hline
    invest                  & Person/ \newline Company & N:N                       & Company                  & 1                          & A person or a company invests into a company.                        \\
    \hline
    guarantee               & Person/ \newline Company & N:N                       & Person/ \newline Company & 1                          & A person or a company guarantees another for some reason like loans. \\
    \hline
    \caption{Description of the data relations.}
    \label{table:relations}
\end{longtable}

{\flushleft \textbf{transfer:}} Fund transfers between accounts. \autoref{table:transfer} shows the attributes.
\begin{table}[H]
    \begin{tabular}{|>{\varNameCell}p{\attributeColumnWidth}|>{\typeCell}p{\typeColumnWidth}|p{\descriptionColumnWidth}|}
        \hline
        \tableHeaderFirst{Attribute} & \tableHeader{Type} & \tableHeader{Description}         \\
        \hline
        timestamp                    & DateTime           & The time when transfer issues.    \\
        \hline
        amount                       & 64-bit Float       & The amount of the transfer.       \\
        \hline
        ordernumber                  & String             & The order number of the transfer. \\
        \hline
        comment                      & String             & The comment of the transfer.      \\
        \hline
        payType                      & String             & The pay type of the transfer.     \\
        \hline
        goodsType                    & String             & The goods type of the transfer.   \\
        \hline
    \end{tabular}
    \caption{Attributes of transfer relation.}
    \label{table:transfer}
\end{table}

{\flushleft \textbf{withdraw:}} Fund is transferred from an account to another of type card. \autoref{table:withdraw} shows the attributes.
\begin{table}[H]
    \begin{tabular}{|>{\varNameCell}p{\attributeColumnWidth}|>{\typeCell}p{\typeColumnWidth}|p{\descriptionColumnWidth}|}
        \hline
        \tableHeaderFirst{Attribute} & \tableHeader{Type} & \tableHeader{Description}      \\
        \hline
        timestamp                    & DateTime           & The time when withdraw issues. \\
        \hline
        amount                       & 64-bit Float       & The amount of the withdraw.    \\
        \hline
    \end{tabular}
    \caption{Attributes of withdraw relation.}
    \label{table:withdraw}
\end{table}

{\flushleft \textbf{repay:}} Loan is repaid from an account. \autoref{table:repay} shows the attributes.
\begin{table}[H]
    \begin{tabular}{|>{\varNameCell}p{\attributeColumnWidth}|>{\typeCell}p{\typeColumnWidth}|p{\descriptionColumnWidth}|}
        \hline
        \tableHeaderFirst{Attribute} & \tableHeader{Type} & \tableHeader{Description}   \\
        \hline
        timestamp                    & DateTime           & The time when repay issues. \\
        \hline
        amount                       & 64-bit Float       & The amount of the repay.    \\
        \hline
    \end{tabular}
    \caption{Attributes of repay relation.}
    \label{table:repay}
\end{table}

{\flushleft \textbf{deposit:}} Loan fund is deposited to an account. \autoref{table:deposit} shows the attributes.
\begin{table}[H]
    \begin{tabular}{|>{\varNameCell}p{\attributeColumnWidth}|>{\typeCell}p{\typeColumnWidth}|p{\descriptionColumnWidth}|}
        \hline
        \tableHeaderFirst{Attribute} & \tableHeader{Type} & \tableHeader{Description}     \\
        \hline
        timestamp                    & DateTime           & The time when deposit issues. \\
        \hline
        amount                       & 64-bit Float       & The amount of the deposit.    \\
        \hline
    \end{tabular}
    \caption{Attributes of deposit relation.}
    \label{table:deposit}
\end{table}

{\flushleft \textbf{signIn:}} An account is signed in with a Media. \autoref{table:signIn} shows the attributes.
\begin{table}[H]
    \begin{tabular}{|>{\varNameCell}p{\attributeColumnWidth}|>{\typeCell}p{\typeColumnWidth}|p{\descriptionColumnWidth}|}
        \hline
        \tableHeaderFirst{Attribute} & \tableHeader{Type} & \tableHeader{Description}     \\
        \hline
        timestamp                    & DateTime           & The time when signIn happens. \\
        \hline
        location                     & String             & The location of the signIn.   \\
        \hline
    \end{tabular}
    \caption{Attributes of signIn relation.}
    \label{table:signIn}
\end{table}

{\flushleft \textbf{invest:}} A person or a company invests in a company. \autoref{table:invest} shows the attributes.
\begin{table}[H]
    \begin{tabular}{|>{\varNameCell}p{\attributeColumnWidth}|>{\typeCell}p{\typeColumnWidth}|p{\descriptionColumnWidth}|}
        \hline
        \tableHeaderFirst{Attribute} & \tableHeader{Type} & \tableHeader{Description}             \\
        \hline
        timestamp                    & DateTime           & The time when the investment happens. \\
        \hline
        ratio                        & 32-bit Float       & The ratio of the investment.          \\
        \hline
    \end{tabular}
    \caption{Attributes of invest relation.}
    \label{table:invest}
\end{table}

{\flushleft \textbf{apply:}} A person or a company applies a Loan. \autoref{table:apply} shows the attributes.
\begin{table}[H]
    \begin{tabular}{|>{\varNameCell}p{\attributeColumnWidth}|>{\typeCell}p{\typeColumnWidth}|p{\descriptionColumnWidth}|}
        \hline
        \tableHeaderFirst{Attribute} & \tableHeader{Type} & \tableHeader{Description}      \\
        \hline
        timestamp                    & DateTime           & The time when apply happens.   \\
        \hline
        organization                 & String             & The organization for the loan. \\
        \hline
    \end{tabular}
    \caption{Attributes of apply relation.}
    \label{table:apply}
\end{table}

{\flushleft \textbf{guarantee:}} A person or a company guarantees another for some reason like Loans. \autoref{table:guarantee} shows the attributes.
\begin{table}[H]
    \begin{tabular}{|>{\varNameCell}p{\attributeColumnWidth}|>{\typeCell}p{\typeColumnWidth}|p{\descriptionColumnWidth}|}
        \hline
        \tableHeaderFirst{Attribute} & \tableHeader{Type} & \tableHeader{Description}                       \\
        \hline
        timestamp                    & DateTime           & The time when guarantee happens.                \\
        \hline
        relationship                 & String             & The relationship between guarantor and applier. \\
        \hline
    \end{tabular}
    \caption{Attributes of guarantee relation.}
    \label{table:guarantee}
\end{table}

{\flushleft \textbf{own:}} A person or a company owns an account. This relation has no attributes.
\begin{table}[H]
    \begin{tabular}{|>{\varNameCell}p{\attributeColumnWidth}|>{\typeCell}p{\typeColumnWidth}|p{\descriptionColumnWidth}|}
        \hline
        \tableHeaderFirst{Attribute} & \tableHeader{Type} & \tableHeader{Description}        \\
        \hline
        timestamp                    & DateTime           & The time when guarantee happens. \\
        \hline
    \end{tabular}
    \caption{Attributes of guarantee relation.}
    \label{table:guarantee}
\end{table}

\section{Data Generation}
 [TODO. This section will be filled after benchmark software designed and developed.]

\section{Output Data}

\subsection{Data Precision}

The datasets are designed and created closely resembling real-world scenarios. {\datagen} produces
financial data having the precision as follows:
\begin{itemize}
    \item The generated 64-bit Float numbers will have precision up to two decimal places for both
          the amount and balance values.
    \item The timestamps are generated with millisecond precision.
\end{itemize}