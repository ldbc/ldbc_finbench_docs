\chapter{Data Definition}
\label{section:data-definition}

\section{Data Types}

\autoref{table:types} describes the different data types used in the benchmark.

\begin{table}[h]
\centering
\begin{tabular}{|>{\typeCell}p{\attributeColumnWidth}|p{\largeDescriptionColumnWidth}|}
    \hline
    \tableHeaderFirst{Type} & \tableHeader{Description} \\
    \hline
    ID & integer type with 64-bit precision. All IDs within a single entity type
    (e.g. Person) are unique, but different entity types (e.g. a Person and an
    Account) might have the same ID.\\
    \hline
    32-bit Integer & integer type with 32-bit precision\\
    \hline
    64-bit Integer & integer type with 64-bit precision\\
    \hline
    String & variable length text of size 40 Unicode characters\\
    \hline
    Long String & variable length text of size 256 Unicode characters\\
    \hline
    Text & variable length text of size 2000 Unicode characters\\
    \hline
    Date & date with a precision of a day, encoded as a string with the
    following format: \textit{yyyy-mm-dd}, where \textit{yyyy} is a four-digit
    integer representing the year, the year, \textit{mm} is a two-digit integer
    representing the month and \textit{dd} is a two-digit integer representing
    the day. \\
    \hline
    DateTime & date with a precision of milliseconds, encoded as a string with
    the following format: \textit{yyyy-mm-ddTHH:MM:ss.sss+0000}, where
    \textit{yyyy} is a four-digit integer representing the year, the year,
    \textit{mm} is a two-digit integer representing the month and \textit{dd} is
    a two-digit integer representing the day, \textit{HH} is a two-digit integer
    representing the hour, \textit{MM} is a two digit integer representing the
    minute and \textit{ss.sss} is a five digit fixed point real number
    representing the seconds up to millisecond precision. Finally, the
    \textit{+0000} of the end represents the timezone, which in this case is
    always GMT.\\
    \hline
    Boolean &  logical type, taking the value of either True of False\\
    \hline
\end{tabular}
\caption{Description of the data types.}
\label{table:types}
\end{table}

\section{Data Schema}

\autoref{figure:schema} shows the data schema in UML. The schema defines the
structure of the data used in the benchmark in terms of entities and their
relations. Data represents a snapshot of the activity in several financial
scenarios during a period of time. The schema specifies different entities,
their attributes, and their relations. All of them are described in the
following sections.

\begin{figure}[htbp]
	\centering
	\includegraphics[width=\linewidth]{figures/data-schema}
	\caption{The \ldbcfinbench data schema}
	\label{figure:schema}
\end{figure}

\subsection{Entities}

{\flushleft \textbf{Person:}} A person of the real world. \autoref{table:person}
shows the attributes.
\begin{table}[H]
    \begin{tabular}{|>{\varNameCell}p{\attributeColumnWidth}|>{\typeCell}p{\typeColumnWidth}|p{\descriptionColumnWidth}|}
        \hline
        \tableHeaderFirst{Attribute} & \tableHeader{Type} &
        \tableHeader{Description} \\
        \hline
        id & ID & The identifier of the person. \\
        \hline
        name & String & The name of the person. \\
        \hline
        isBlocked & Boolean & If the person is blocked or concerned in systems. \\
        \hline
    \end{tabular}
    \caption{Attributes of Person entity.}
    \label{table:person}
\end{table}

{\flushleft \textbf{Company:}} A company of the real world, which persons work
in and persons invest. \autoref{table:company} shows the attributes.
\begin{table}[H]
    \begin{tabular}{|>{\varNameCell}p{\attributeColumnWidth}|>{\typeCell}p{\typeColumnWidth}|p{\descriptionColumnWidth}|}
        \hline
        \tableHeaderFirst{Attribute} & \tableHeader{Type} &
        \tableHeader{Description} \\
        \hline
        id & ID & The identifier of the company. \\
        \hline
        name & String & The name of the company. \\
        \hline
        isBlocked & Boolean & If the company is blocked or concerned in systems. \\
        \hline
    \end{tabular}
    \caption{Attributes of Company entity.}
    \label{table:company}
\end{table}

{\flushleft \textbf{Account:}} An account in real world financial systems, which
is registered and owned by persons and companies. It includes many types such as
personalDeposit, personalCredit, etc. It can deal with other accounts.
\autoref{table:account} shows the attributes.
\begin{table}[H]
    \begin{tabular}{|>{\varNameCell}p{\attributeColumnWidth}|>{\typeCell}p{\typeColumnWidth}|p{\descriptionColumnWidth}|}
        \hline
        \tableHeaderFirst{Attribute} & \tableHeader{Type} &
        \tableHeader{Description} \\
        \hline
        id & ID & The identifier of the account. \\
        \hline
        createTime & DateTime & The time when the account created. \\
        \hline
        isBlocked & Boolean & If the account is blocked or concerned in systems. \\
        \hline
        type & String & The type of Account including personalDeposit,
        personalCredit, companyDeposit, card. \\
        \hline
    \end{tabular}
    \caption{Attributes of Account entity.}
    \label{table:account}
\end{table}

{\flushleft \textbf{Loan:}} A loan for persons and company to apply in real
world. \autoref{table:loan} shows the attributes.
\begin{table}[H]
    \begin{tabular}{|>{\varNameCell}p{\attributeColumnWidth}|>{\typeCell}p{\typeColumnWidth}|p{\descriptionColumnWidth}|}
        \hline
        \tableHeaderFirst{Attribute} & \tableHeader{Type} &
        \tableHeader{Description} \\
        \hline
        id & ID & The identifier of the loan. \\
        \hline
        loanAmount & 64-bit Integer & The amount of a loan. \\
        \hline
        balance & 64-bit Integer & The balance of a loan. \\
        \hline
    \end{tabular}
    \caption{Attributes of Loan entity.}
    \label{table:loan}
\end{table}

{\flushleft \textbf{Medium:}} An abstract standing for things that users use to
sign in account in real world, such as IP, mac, phone numbers.
\autoref{table:medium} shows the attributes.
\begin{table}[H]
    \begin{tabular}{|>{\varNameCell}p{\attributeColumnWidth}|>{\typeCell}p{\typeColumnWidth}|p{\descriptionColumnWidth}|}
        \hline
        \tableHeaderFirst{Attribute} & \tableHeader{Type} &
        \tableHeader{Description} \\
        \hline
        id & ID & The identifier of the medium. \\
        \hline
        name & String & The name of the medium. \\
        \hline
        isBlocked & Boolean & If the medium is blocked or concerned in systems. \\
        \hline
    \end{tabular}
    \caption{Attributes of Medium entity.}
    \label{table:medium}
\end{table}

\subsection{Relations}
Relations connect entities of different types showed in \autoref{table:relations}.
Except that own and workIn has no attributes, the attributes of other relations 
are showed in the following tables.

\begin{longtable}{|>{\varNameCell}p{1.5cm}|>{\typeCell}p{2.5cm}|>{\typeCell}p{2.5cm}|>{\edgeDirectionCell}p{0.5cm}|p{6cm}|}
        \hline
        \tableHeaderFirst{Name} & \tableHeader{Tail} & \tableHeader{Head} & \tableHeader{Multiplicity} & \tableHeader{Description} \\
        \hline
        signIn & Medium & Account & N & An account is signed in with a Media. \\
        \hline
        own & Person/Company & Account & 1 & A person or a company owns an account. \\
        \hline
        transfer & Account & Account & N & Fund transfers between two accounts.\\
        \hline
        deposit & Loan & Account & N & Loan fund is deposited to an account. \\
        \hline
        repay & Account & Loan & N & Loan is repaid from an account. \\
        \hline
        withdraw & Account & Account & N & Fund is transferred from an account to another account whose type is card. \\
        \hline
        invest & Person/Company & Company & 1 & A person or a company invests a company. \\
        \hline
        workIn & Person & Company & 1 & A person works in a company. \\
        \hline
        apply & Person/Company & Loan & 1 & A person or a company applies a Loan. \\
        \hline
        guarantee & Person/Company & Person/Company & 1 & A person or a company guarantees another for some reason like loans. \\
        \hline
        \caption{Description of the data relations.}
        \label{table:relations}
\end{longtable}

{\flushleft \textbf{transfer:}} Fund transfers between accounts. \autoref{table:transfer} shows the attributes.
\begin{table}[H]
    \begin{tabular}{|>{\varNameCell}p{\attributeColumnWidth}|>{\typeCell}p{\typeColumnWidth}|p{\descriptionColumnWidth}|}
        \hline
        \tableHeaderFirst{Attribute} & \tableHeader{Type} & \tableHeader{Description} \\
        \hline
        timestamp & DateTime & The time when transfer issues.\\
        \hline
        amount & Long & The amount of the transfer.\\
        \hline
        type & String & The type of the transfer. \\
        \hline
    \end{tabular}
    \caption{Attributes of transfer relation.}
    \label{table:transfer}
\end{table}

{\flushleft \textbf{withdraw:}} Fund is transferred from an account to another of type card. \autoref{table:withdraw} shows the attributes.
\begin{table}[H]
    \begin{tabular}{|>{\varNameCell}p{\attributeColumnWidth}|>{\typeCell}p{\typeColumnWidth}|p{\descriptionColumnWidth}|}
        \hline
        \tableHeaderFirst{Attribute} & \tableHeader{Type} & \tableHeader{Description} \\
        \hline
        timestamp & DateTime & The time when withdraw issues. \\
        \hline
        amount & Long & The amount of the withdraw. \\
        \hline
    \end{tabular}
    \caption{Attributes of withdraw relation.}
    \label{table:withdraw}
\end{table}

{\flushleft \textbf{repay:}} Loan is repaid from an account. \autoref{table:repay} shows the attributes.
\begin{table}[H]
    \begin{tabular}{|>{\varNameCell}p{\attributeColumnWidth}|>{\typeCell}p{\typeColumnWidth}|p{\descriptionColumnWidth}|}
        \hline
        \tableHeaderFirst{Attribute} & \tableHeader{Type} & \tableHeader{Description} \\
        \hline
        timestamp & DateTime & The time when repay issues. \\
        \hline
        amount & Long & The amount of the repay. \\
        \hline
    \end{tabular}
    \caption{Attributes of repay relation.}
    \label{table:repay}
\end{table}

{\flushleft \textbf{deposit:}} Loan fund is deposited to an account. \autoref{table:deposit} shows the attributes.
\begin{table}[H]
    \begin{tabular}{|>{\varNameCell}p{\attributeColumnWidth}|>{\typeCell}p{\typeColumnWidth}|p{\descriptionColumnWidth}|}
        \hline
        \tableHeaderFirst{Attribute} & \tableHeader{Type} & \tableHeader{Description} \\
        \hline
        timestamp & DateTime & The time when deposit issues. \\
        \hline
        amount & Long & The amount of the deposit. \\
        \hline
    \end{tabular}
    \caption{Attributes of deposit relation.}
    \label{table:deposit}
\end{table}

{\flushleft \textbf{signIn:}} An account is signed in with a Media. \autoref{table:signIn} shows the attributes.
\begin{table}[H]
    \begin{tabular}{|>{\varNameCell}p{\attributeColumnWidth}|>{\typeCell}p{\typeColumnWidth}|p{\descriptionColumnWidth}|}
        \hline
        \tableHeaderFirst{Attribute} & \tableHeader{Type} & \tableHeader{Description} \\
        \hline
        timestamp & DateTime & The time when signIn happens. \\
        \hline
    \end{tabular}
    \caption{Attributes of signIn relation.}
    \label{table:signIn}
\end{table}

{\flushleft \textbf{invest:}} A person or a company invests a company. \autoref{table:invest} shows the attributes.
\begin{table}[H]
    \begin{tabular}{|>{\varNameCell}p{\attributeColumnWidth}|>{\typeCell}p{\typeColumnWidth}|p{\descriptionColumnWidth}|}
        \hline
        \tableHeaderFirst{Attribute} & \tableHeader{Type} & \tableHeader{Description} \\
        \hline
        timestamp & DateTime & The time when invest happens. \\
        \hline
    \end{tabular}
    \caption{Attributes of invest relation.}
    \label{table:invest}
\end{table}

{\flushleft \textbf{apply:}} A person or a company applies a Loan. \autoref{table:apply} shows the attributes.
\begin{table}[H]
    \begin{tabular}{|>{\varNameCell}p{\attributeColumnWidth}|>{\typeCell}p{\typeColumnWidth}|p{\descriptionColumnWidth}|}
        \hline
        \tableHeaderFirst{Attribute} & \tableHeader{Type} & \tableHeader{Description} \\
        \hline
        timestamp & DateTime & The time when apply happens. \\
        \hline
    \end{tabular}
    \caption{Attributes of apply relation.}
    \label{table:apply}
\end{table}

{\flushleft \textbf{guarantee:}} A person or a company guarantees another for some reason like Loans. \autoref{table:guarantee} shows the attributes.
\begin{table}[H]
    \begin{tabular}{|>{\varNameCell}p{\attributeColumnWidth}|>{\typeCell}p{\typeColumnWidth}|p{\descriptionColumnWidth}|}
        \hline
        \tableHeaderFirst{Attribute} & \tableHeader{Type} & \tableHeader{Description} \\
        \hline
        timestamp & DateTime & The time when guarantee happens. \\
        \hline
    \end{tabular}
    \caption{Attributes of guarantee relation.}
    \label{table:guarantee}
\end{table}

\section{Data Generation}
[TODO. This section will be filled after benchmark software designed and developed.]

\section{Output Data}
[TODO. This section will be filled after benchmark software designed and developed.]\