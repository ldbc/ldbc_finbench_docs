\cpSection[2.3]{2.1}{QOPT}{Rich join order optimization}

This choke point tests the ability of the query optimizer to find optimal join orders. A graph can be traversed in different ways. In the relational model, this is equivalent to different join orders.
The execution time of these orders may differ by orders of magnitude. Therefore, finding an efficient join (traversal) order is important, which in general, requires enumeration of all the possibilities.
The enumeration is complicated by operators that are not freely re-orderable like semi-, \mbox{anti-,} and outer-joins. Because of this difficulty most join enumeration algorithms do not enumerate all possible plans, and therefore can miss the optimal join order. Therefore, this choke point tests the ability of the query optimizer to find optimal join (traversal) orders.

%%%%%%%%%%%%%%%%%%%%%%%%%%%%%%%%%%%%%%%%%%%%%%%%%%%%%%%%%%%%%%%%%%%%%%%%%%%%%%

\paragraph{Queries}
{\raggedright
    % \queryRefCard{bi-read-03}{BI}{3}
    % \queryRefCard{bi-read-04}{BI}{4}
    % \queryRefCard{bi-read-08}{BI}{8}
    % \queryRefCard{bi-read-13}{BI}{13}
    % \queryRefCard{bi-read-14}{BI}{14}
    % \queryRefCard{bi-read-15}{BI}{15}
    % \queryRefCard{bi-read-17}{BI}{17}
    % \queryRefCard{interactive-complex-read-01}{IC}{1}
    % \queryRefCard{interactive-complex-read-03}{IC}{3}

}