\snbCPSection{8.6}{LANG}{Handling paths}

\paragraph{Description.}

Handling paths as first-class citizens is one of the key distinguishing features
of graph database systems~\cite{DBLP:conf/sigmod/AnglesABBFGLPPS18}. Hence,
additionally to reachability-style checks, a language should be able to express
queries that operate on elements of a path, \eg calculate a score on each edge
of the path. Also, some use cases specify uniqueness constraints on
paths~\cite{DBLP:journals/csur/AnglesABHRV17}: \emph{arbitrary path},
\emph{shortest path}, \emph{no-repeated-node semantics} (also known as
\emph{simple paths}), and \emph{no-repeated-edge semantics} (also known as
\emph{trails}). Other variants are also used in rare cases, such as
\emph{maximal} (non-expandable) or \emph{minimal} (non-contractable) paths.

\paragraph{Note on terminology.}

The \emph{Glossary of graph theory terms} page of
Wikipedia\footnote{\url{https://en.wikipedia.org/wiki/Glossary_of_graph_theory_terms}}
defines \emph{paths} as follows: ``A path may either be a walk (a sequence of
nodes and edges, with both endpoints of an edge appearing adjacent to it in the
sequence) or a simple path (a walk with no repetitions of nodes or edges),
depending on the source.'' In this work, we use the first definition, which is
more common in modern graph database systems and is also followed in a recent
survey on graph query languages~\cite{DBLP:journals/csur/AnglesABHRV17}.

%%%%%%%%%%%%%%%%%%%%%%%%%%%%%%%%%%%%%%%%%%%%%%%%%%%%%%%%%%%%%%%%%%%%%%%%%%%%%

\paragraph{Queries}
{\raggedright
    % \queryRefCard{interactive-complex-read-14}{IC}{14}
}