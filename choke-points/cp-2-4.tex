\cpSection[2.2]{2.4}{QOPT}{Sparse foreign key joins}

This choke point tests the performance of join operators when the join is sparse. Sometimes joins involve relations where only a small percentage of rows in one of the tables is required to satisfy a join. When tables are larger, typical join methods can be sub-optimal. Partitioning the sparse table, using Hash Clustered indices or implementing Bloom-filter tests inside the join are techniques to improve the performance in such situations~\cite{DBLP:journals/csur/Graefe93}.

%%%%%%%%%%%%%%%%%%%%%%%%%%%%%%%%%%%%%%%%%%%%%%%%%%%%%%%%%%%%%%%%%%%%%%%%%%%%%%

\paragraph{Queries}
{\raggedright
    % \queryRefCard{bi-read-02}{BI}{2}
    % \queryRefCard{bi-read-03}{BI}{3}
    % \queryRefCard{bi-read-04}{BI}{4}
    % \queryRefCard{bi-read-10}{BI}{10}
    % \queryRefCard{bi-read-13}{BI}{13}
    % \queryRefCard{bi-read-15}{BI}{15}
    % \queryRefCard{interactive-complex-read-08}{IC}{8}
    % \queryRefCard{interactive-complex-read-11}{IC}{11}

}