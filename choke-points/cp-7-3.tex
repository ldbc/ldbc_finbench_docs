\cpSection{7.3}{QEXE}{Execution of a transitive step}

This choke point tests the ability of the query execution engine to efficiently execute transitive steps. Graph workloads may have transitive operations, for example finding a shortest path between nodes.
This involves repeated execution of a short lookup, often on many values at the
same time, while usually having an end condition, \eg the target node being reached or having reached the border of a search going in the opposite direction.
For the best efficiency, these operations can be merged or tightly coupled to
the index operations themselves. Also parallelization may be possible but may
need to deal with a global state, \eg set of visited nodes.
There are many possible tradeoffs between generality and performance.

%%%%%%%%%%%%%%%%%%%%%%%%%%%%%%%%%%%%%%%%%%%%%%%%%%%%%%%%%%%%%%%%%%%%%%%%%%%%%%

\paragraph{Queries}
{\raggedright
    % \queryRefCard{bi-read-09}{BI}{9}
    % \queryRefCard{bi-read-10}{BI}{10}
    % \queryRefCard{bi-read-15}{BI}{15}
    % \queryRefCard{interactive-complex-read-12}{IC}{12}
    % \queryRefCard{interactive-complex-read-13}{IC}{13}
    % \queryRefCard{interactive-complex-read-14}{IC}{14}

}