\snbCPSection{7.2}{QOPT}{Cardinality estimation of transitive paths}

This choke point tests the ability of the query optimizer to properly estimate
the cardinality of intermediate results when executing transitive paths. A
transitive path may occur in a ``fact table'' or a ``dimension table'' position.
A transitive path may cover a tree or a graph, \eg descendants in a geographical
hierarchy \vs graph neighborhood or transitive closure in a many-to-many
connected social network. In order to decide proper join order and type, the
cardinality of the expansion of the transitive path needs to be correctly
estimated. This could for example take the form of executing on a sample of the
data in the cost model or of gathering special statistics, \eg the depth and
fan-out of a tree. In the case of hierarchical dimensions, \eg geographic
locations or other hierarchical classifications, detecting the cardinality of
the transitive path will allow one to go to a star schema plan with scan of a
fact table with a selective hash join. Such a plan will be on the other hand
very bad for example if the hash table is much larger than the ``fact table''
being scanned.

%%%%%%%%%%%%%%%%%%%%%%%%%%%%%%%%%%%%%%%%%%%%%%%%%%%%%%%%%%%%%%%%%%%%%%%%%%%%%%

\paragraph{Queries}
{\raggedright
    % \queryRefCard{interactive-complex-read-14}{IC}{14}
}