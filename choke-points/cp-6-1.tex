\tpcCPSection[6.3]{6.1}{QEXE}{Inter-query result reuse}

This choke point tests the ability of the query execution engine to reuse
results from different queries. Sometimes with a high number of streams a
significant amount of identical queries emerge in the resulting workload. The
reason is that certain parameters, as generated by the workload generator, have
only a limited amount of parameters bindings. This weakness opens up the
possibility of using a query result cache, to eliminate the repetitive part of
the workload. A further opportunity that detects even more overlap is the work
on recycling, which does not only cache final query results, but also
intermediate query results of a ``high worth''. Here, worth is a combination of
partial-query result size, partial-query evaluation cost, and observed (or
estimated) frequency of the partial-query in the workload.

%%%%%%%%%%%%%%%%%%%%%%%%%%%%%%%%%%%%%%%%%%%%%%%%%%%%%%%%%%%%%%%%%%%%%%%%%%%%%%

\IfFileExists{choke-point-query-mapping/cp-6-1}{\tpcCPSection[6.3]{6.1}{QEXE}{Inter-query result reuse}

This choke point tests the ability of the query execution engine to reuse
results from different queries. Sometimes with a high number of streams a
significant amount of identical queries emerge in the resulting workload. The
reason is that certain parameters, as generated by the workload generator, have
only a limited amount of parameters bindings. This weakness opens up the
possibility of using a query result cache, to eliminate the repetitive part of
the workload. A further opportunity that detects even more overlap is the work
on recycling, which does not only cache final query results, but also
intermediate query results of a ``high worth''. Here, worth is a combination of
partial-query result size, partial-query evaluation cost, and observed (or
estimated) frequency of the partial-query in the workload.

%%%%%%%%%%%%%%%%%%%%%%%%%%%%%%%%%%%%%%%%%%%%%%%%%%%%%%%%%%%%%%%%%%%%%%%%%%%%%%

\IfFileExists{choke-point-query-mapping/cp-6-1}{\tpcCPSection[6.3]{6.1}{QEXE}{Inter-query result reuse}

This choke point tests the ability of the query execution engine to reuse
results from different queries. Sometimes with a high number of streams a
significant amount of identical queries emerge in the resulting workload. The
reason is that certain parameters, as generated by the workload generator, have
only a limited amount of parameters bindings. This weakness opens up the
possibility of using a query result cache, to eliminate the repetitive part of
the workload. A further opportunity that detects even more overlap is the work
on recycling, which does not only cache final query results, but also
intermediate query results of a ``high worth''. Here, worth is a combination of
partial-query result size, partial-query evaluation cost, and observed (or
estimated) frequency of the partial-query in the workload.

%%%%%%%%%%%%%%%%%%%%%%%%%%%%%%%%%%%%%%%%%%%%%%%%%%%%%%%%%%%%%%%%%%%%%%%%%%%%%%

\IfFileExists{choke-point-query-mapping/cp-6-1}{\tpcCPSection[6.3]{6.1}{QEXE}{Inter-query result reuse}

This choke point tests the ability of the query execution engine to reuse
results from different queries. Sometimes with a high number of streams a
significant amount of identical queries emerge in the resulting workload. The
reason is that certain parameters, as generated by the workload generator, have
only a limited amount of parameters bindings. This weakness opens up the
possibility of using a query result cache, to eliminate the repetitive part of
the workload. A further opportunity that detects even more overlap is the work
on recycling, which does not only cache final query results, but also
intermediate query results of a ``high worth''. Here, worth is a combination of
partial-query result size, partial-query evaluation cost, and observed (or
estimated) frequency of the partial-query in the workload.

%%%%%%%%%%%%%%%%%%%%%%%%%%%%%%%%%%%%%%%%%%%%%%%%%%%%%%%%%%%%%%%%%%%%%%%%%%%%%%

\IfFileExists{choke-point-query-mapping/cp-6-1}{\input{choke-point-query-mapping/cp-6-1}}{}
}{}
}{}
}{}
