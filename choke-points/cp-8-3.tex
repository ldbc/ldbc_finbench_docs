\snbCPSection{8.3}{LANG}{Ranking-style queries}

\paragraph{Description.}

Along with aggregations, BI workloads often use \emph{window functions}, which
perform aggregations without grouping input tuples to a single output tuple.  A
common use case for windowing is \emph{ranking}, \ie selecting the top element
with additional values in the tuple (nodes, edges or
attributes).\footnote{PostgreSQL defines the \lstinline[language=sql]{OVER}
keyword to use aggregation functions as window functions, and the
\lstinline[language=sql]{rank()} function to produce numerical ranks, see
\url{https://www.postgresql.org/docs/9.1/static/tutorial-window.html} for
details.}

%%%%%%%%%%%%%%%%%%%%%%%%%%%%%%%%%%%%%%%%%%%%%%%%%%%%%%%%%%%%%%%%%%%%%%%%%%%%%%

\paragraph{Queries}
{\raggedright
    % \queryRefCard{interactive-complex-read-14}{IC}{14}
}