\newCPSection{8.8}{LANG}{Recursive path filtering pattern}

Sometimes when tracing a fund flow, such a pattern is expected that find a path
with recursive filters. For example, filters are expected to assume a path A
-[${e_1}$]-> B -[${e_2}$]-> ... -> X.

\begin{itemize}
    \item The timestamp order: ${e_1}$ < ${e_2}$ < … < ${e_i}$
    \item The amount order: ${e_1}$ > ${e_2}$ > … > ${e_i}$
    \item The time window: ${e_{i-1}}$ < ${e_i}$ < ${e_{i-1}}$ + $\vec{\Delta}$,
    $\vec{\Delta}$ is a given constant.
\end{itemize}

Such queries that require "all timestamps in the transfer trace are in ascending order" or the "upstream" edge are
difficult to explain in plain Cypher (or GQL or SQL/PGQ) because they require support for the category of queries
"Regular expression with memory" as described in this paper\cite{10.1145/2274576.2274585}. Another possible solution is
adding keywords like \emph{SEQUENTIAL} and \emph{DELTA} referring to an extension of \emph{Cypher}~\cite{tcypher}. 

%%%%%%%%%%%%%%%%%%%%%%%%%%%%%%%%%%%%%%%%%%%%%%%%%%%%%%%%%%%%%%%%%%%%%%%%%%%%%

\IfFileExists{choke-point-query-mapping/cp-8-8}{\newCPSection{8.8}{LANG}{Recursive path filtering pattern}

Sometimes when tracing a fund flow, such pattern is expected that find a path with recursive filters.
For example, filters are expected assuming a path A -[${e_1}$]-> B -[${e_2}$]-> ... -> X.

\begin{itemize}
    \item The timestamp order: ${e_1}$ < ${e_2}$ < … < ${e_i}$
    \item The amount order: ${e_1}$ > ${e_2}$ > … > ${e_i}$
    \item The time window: ${e_{i-1}}$ < ${e_i}$ < ${e_{i-1}}$ + $\vec{\Delta}$, $\vec{\Delta}$ is a given constant.
\end{itemize}


The feature that supporting backward recursive filtering based on edge properties is expected. A possible solution is adding keywords like \emph{SEQUENTIAL} and \emph{DELTA} referring to an extension of \emph{Cypher}\cite{tcypher}.

%%%%%%%%%%%%%%%%%%%%%%%%%%%%%%%%%%%%%%%%%%%%%%%%%%%%%%%%%%%%%%%%%%%%%%%%%%%%%

\paragraph{Queries}
{\raggedright
    % \queryRefCard{interactive-complex-read-14}{IC}{14}

}}{}
