\cpSection{2.3}{QOPT}{Join type selection}

This choke point tests the ability of the query optimizer to select the proper join operator type, which implies accurate estimates of cardinalities.
Depending on the cardinalities of both sides of a join, a hash or an index-based join operator is more appropriate.
This is especially important with column stores, where one usually has an index on everything. Deciding to use a hash join requires a good estimation of cardinalities on both the probe and build sides.
In TPC-H, the use of hash join is almost a foregone conclusion in many cases, since an implementation will usually not even define an index on foreign key columns.
There is a break even point between index and hash based plans, depending on the cardinality on the probe and build sides.

%%%%%%%%%%%%%%%%%%%%%%%%%%%%%%%%%%%%%%%%%%%%%%%%%%%%%%%%%%%%%%%%%%%%%%%%%%%%%%

\paragraph{Queries}
{\raggedright
    % \queryRefCard{bi-read-04}{BI}{4}
    % \queryRefCard{bi-read-05}{BI}{5}
    % \queryRefCard{bi-read-06}{BI}{6}
    % \queryRefCard{bi-read-08}{BI}{8}
    % \queryRefCard{bi-read-09}{BI}{9}
    % \queryRefCard{bi-read-10}{BI}{10}
    % \queryRefCard{bi-read-11}{BI}{11}
    % \queryRefCard{bi-read-13}{BI}{13}
    % \queryRefCard{bi-read-17}{BI}{17}
    % \queryRefCard{interactive-complex-read-02}{IC}{2}
    % \queryRefCard{interactive-complex-read-04}{IC}{4}
    % \queryRefCard{interactive-complex-read-05}{IC}{5}
    % \queryRefCard{interactive-complex-read-07}{IC}{7}
    % \queryRefCard{interactive-complex-read-09}{IC}{9}
    % \queryRefCard{interactive-complex-read-10}{IC}{10}
    % \queryRefCard{interactive-complex-read-11}{IC}{11}

}