\chapter{Introduction}
\label{sec:introduction}

%%%%%%%%%%%%%%%%%%%%%%%%%%%%%%%%%%%%%%%%%%%%%%%%%%%%%%%%%%%%%%%%%%%%%%%%%%%%%%
%%%%%%%%%%%%%%%%%%%%%%%%%%%%%%%%%%%%%%%%%%%%%%%%%%%%%%%%%%%%%%%%%%%%%%%%%%%%%%
%%%%%%%%%%%%%%%%%%%%%%%%%%%%%%%%%%%%%%%%%%%%%%%%%%%%%%%%%%%%%%%%%%%%%%%%%%%%%%

\section{Practice basis for FinBench}

Based on a comprehensive survey on financial scenarios inside and outside
AntGroup, we summarize some common user cases in risk control, AML (Anti-Money
Laundering), KYC(Know Your Customer), and so on. With the best practices in the
industry, we propose this design for LDBC Financial Benchmark.

%%%%%%%%%%%%%%%%%%%%%%%%%%%%%%%%%%%%%%%%%%%%%%%%%%%%%%%%%%%%%%%%%%%%%%%%%%%%%%
%%%%%%%%%%%%%%%%%%%%%%%%%%%%%%%%%%%%%%%%%%%%%%%%%%%%%%%%%%%%%%%%%%%%%%%%%%%%%%
%%%%%%%%%%%%%%%%%%%%%%%%%%%%%%%%%%%%%%%%%%%%%%%%%%%%%%%%%%%%%%%%%%%%%%%%%%%%%%

\section{Design concepts of FinBench}

\subsection{Data Schema}

The data design is based on a review of actual data in systems. The data schema
for FinBench is inspired by the real data in financial systems. Stored data in
systems contain entities in the real world including accounts, medium(device,
IP, etc.), persons, companies, orders, loans, and so on. These entities are
Nodes in the data schema. And the edges in the data schema reflect financial
actions in the real world like transferring funds from one account to another,
guaranteeing by one person for another. The designed data schema is specified in
Section 2.3.

\subsection{Load definition}

In this draft, we do not finish the load design with the workloads. But we
conclude some patterns of load after reviewing audit logs of systems. They are:
\begin{itemize}
    \item Data read and write intermittently with random intervals.
    \item There are some light and extremely heavy loads periodically.
    \item Large scale data ETLs are triggered when at midnight or some time the
    system load is light.
    \item Tight latency constraints.
\end{itemize}

\subsection{Workloads}

Compared with LDBC SNB, we divide the queries into three workloads based on
their complexity and application latency. They are:
\begin{itemize}
    \item Online Workload (See Section 4). This workload is supposed to
    include cases in online applications which are expected to be finished in
    low latency. The latency may range from tens to hundreds in millisecond
    precision. The cases are usually accessing at most 3 step neighborhood from
    a start node. In this draft, we fill cases in the online workload for
    discussion.
    \item Nearline Workload (See Section 5). This workload is supposed to
    include cases in nearline applications which are expected to be finished in
    higher latency than online applications but lower latency than offline
    applications. The latency may range from several seconds to several minutes.
    The cases usually include subgraph traversal, pattern matching, and deeper
    neighborhood accessing. Nearline workload will be designed in the future.
    \item Offline Workload (See Section 6). This workload is supposed to
    include cases in offline applications which are expected to be finished in
    high latency. The latency may range from tens of minutes and even many
    hours. The cases are usually performing iterative graph analytics. Offline
    workload will be designed in the future.
\end{itemize}
In Online workload, the queries include read queries, write queries, and
read-write queries. Read-write query is a significant design that reflects the
complexity of financial systems. In real-time risk control, risk analysis for
involved accounts is supposed to be triggered intermediately when some deals
related are recorded. Abstracting from such scenarios, we propose a concept
Read-write Query. A read-write query is composed of read queries and write
queries consecutively. For example, a read-write query is composed of inserting
transfer edges(a write query), risk analysis for a specific account(a read
query), in the example mentioned above. For detail, see Section 4.3.
%%%%%%%%%%%%%%%%%%%%%%%%%%%%%%%%%%%%%%%%%%%%%%%%%%%%%%%%%%%%%%%%%%%%%%%%%%%%%%
%%%%%%%%%%%%%%%%%%%%%%%%%%%%%%%%%%%%%%%%%%%%%%%%%%%%%%%%%%%%%%%%%%%%%%%%%%%%%%
%%%%%%%%%%%%%%%%%%%%%%%%%%%%%%%%%%%%%%%%%%%%%%%%%%%%%%%%%%%%%%%%%%%%%%%%%%%%%%

\section{Differences between FinBench and SNB}

We highlight several differences between FinBench and SNB as the following:
\begin{itemize}
    \item Multiple edges can exist between two vertices, e.g., many
    money-transfers can occur between two accounts each day.
    \item Read-write queries, which is a query sequence with a mix of read and
    write queries.
    \item Filtering with backward dependency in variable-length paths, e.g.,
    finding all money-transfer paths A ->[e1]->B-[e2]->...->X  in which the
    timestamp of each transfer ei  is larger than that of ei-1
    \item Property update queries, e.g., marking an account as blocked or
    high-risk for risk control.
    \item Latency sensitive, e.g., some queries need to return in less than
    10ms.
\end{itemize}

%%%%%%%%%%%%%%%%%%%%%%%%%%%%%%%%%%%%%%%%%%%%%%%%%%%%%%%%%%%%%%%%%%%%%%%%%%%%%%
%%%%%%%%%%%%%%%%%%%%%%%%%%%%%%%%%%%%%%%%%%%%%%%%%%%%%%%%%%%%%%%%%%%%%%%%%%%%%%
%%%%%%%%%%%%%%%%%%%%%%%%%%%%%%%%%%%%%%%%%%%%%%%%%%%%%%%%%%%%%%%%%%%%%%%%%%%%%%