\chapter{Introduction}
\label{sec:introduction}

%%%%%%%%%%%%%%%%%%%%%%%%%%%%%%%%%%%%%%%%%%%%%%%%%%%%%%%%%%%%%%%%%%%%%%%%%%%%%%
%%%%%%%%%%%%%%%%%%%%%%%%%%%%%%%%%%%%%%%%%%%%%%%%%%%%%%%%%%%%%%%%%%%%%%%%%%%%%%
%%%%%%%%%%%%%%%%%%%%%%%%%%%%%%%%%%%%%%%%%%%%%%%%%%%%%%%%%%%%%%%%%%%%%%%%%%%%%%

\section{Motivation}

Inspired by \ldbcsnb, a task force proposed by AntGroup~\cite{antgroup} is
formed by the principal actors in the field of financial graph-like data
management with help from LDBC to design a new benchmark, \ldbcfinbench (LDBC's
Financial Benchmark). The task force intends to define a framework which is more
applicable to financial scenarios to fairly test and compare different graph
based technologies. To this end, they carefully design the dataset and workload
using  their rich practical experience as members of the financial industry. \ldbcfinbench
is distinguished and characterized by the special features and patterns in
financial industry.

%%%%%%%%%%%%%%%%%%%%%%%%%%%%%%%%%%%%%%%%%%%%%%%%%%%%%%%%%%%%%%%%%%%%%%%%%%%%%%
%%%%%%%%%%%%%%%%%%%%%%%%%%%%%%%%%%%%%%%%%%%%%%%%%%%%%%%%%%%%%%%%%%%%%%%%%%%%%%
%%%%%%%%%%%%%%%%%%%%%%%%%%%%%%%%%%%%%%%%%%%%%%%%%%%%%%%%%%%%%%%%%%%%%%%%%%%%%%

\section{Relevance to the Industry}

\ldbcfinbench is intended to provide the following value to these relevant
stakeholders:

\begin{itemize}
      \item For \textbf{users} facing graph processing tasks in financial industry,
            \ldbcfinbench provides a recognizable scenario against which it is possible
            to compare merits of different products and technologies. By covering a wide
            variety of scales and price points, \ldbcfinbench can serve as an aid to
            technology selection.
      \item For \textbf{vendors} of graph database technology, \ldbcfinbench provides a
            checklist of features and performance characteristics that helps in product
            positioning and can serve to guide new development.
      \item For \textbf{researchers}, both industrial and academic, the \ldbcfinbench
            dataset and workload provide interesting challenges in multiple choke point
            areas, and help compare the efficiency of existing technology in these
            areas.
\end{itemize}

The technological scope of \ldbcfinbench comprises all systems that one might
conceivably use to perform financial data management tasks including
\textbf{Graph database management systems} (\eg Neo4j, TuGraph, etc.), \textbf{
      Graph processing frameworks} (\eg Giraph, Ligra, etc.), \textbf{RDF database
      systems} (\eg Virtuoso, AWS Neptune, etc.), \textbf{Relational database systems}
(\eg MySQL, Oracle, etc.), \textbf{NoSQL database systems} (\eg key-value stores
such as HBase, Redis, MongoDB, CouchDB, or even MapReduce systems like Hadoop
and Pig).

%%%%%%%%%%%%%%%%%%%%%%%%%%%%%%%%%%%%%%%%%%%%%%%%%%%%%%%%%%%%%%%%%%%%%%%%%%%%%%
%%%%%%%%%%%%%%%%%%%%%%%%%%%%%%%%%%%%%%%%%%%%%%%%%%%%%%%%%%%%%%%%%%%%%%%%%%%%%%
%%%%%%%%%%%%%%%%%%%%%%%%%%%%%%%%%%%%%%%%%%%%%%%%%%%%%%%%%%%%%%%%%%%%%%%%%%%%%%

\section{Participation of Industry and Academia}

Initially, the \ldbcfinbench task force is formed by relevant actors mainly from
industry. In the process of design and development, we also received supports and
suggestions from fellows in academia. All the participants have contributed with
their experience and expertise to make this benchmark a credible effort. The list
of participants is as follows.

\begin{itemize}
  \item AntGroup (entity)
  \item CreateLink (entity)
  \item Ultipa (entity)
  \item StarGraph (entity)
  \item Vesoft (entity)
  \item Pometry (entity)
  \item Katana (entity)
  \item Intel (entity)
  \item TigerGraph (entity)
  \item Koji Annoura (individual)
\end{itemize}

%%%%%%%%%%%%%%%%%%%%%%%%%%%%%%%%%%%%%%%%%%%%%%%%%%%%%%%%%%%%%%%%%%%%%%%%%%%%%%
%%%%%%%%%%%%%%%%%%%%%%%%%%%%%%%%%%%%%%%%%%%%%%%%%%%%%%%%%%%%%%%%%%%%%%%%%%%%%%
%%%%%%%%%%%%%%%%%%%%%%%%%%%%%%%%%%%%%%%%%%%%%%%%%%%%%%%%%%%%%%%%%%%%%%%%%%%%%%

\section{Software Components}
\label{sec:software-components}

The source code of this specification and the benchmark suite are available
open-source:
\begin{itemize}
      \item \ldbcfinbench Specification: \url{https://github.com/ldbc/ldbc_finbench_docs}
      \item \ldbcfinbench Data Generator: \url{https://github.com/ldbc/ldbc_finbench_datagen}
      \item \ldbcfinbench Driver: \url{https://github.com/ldbc/ldbc_finbench_driver}
      \item Transaction Workload Implementation: \url{https://github.com/ldbc/ldbc_finbench_transaction_impls}
      \item Analytics Workload: future work
\end{itemize}

Note that the \texttt{main} branch for these repositories is under development
by default. Please refer to the releases and branch started with \texttt{v} and
named \texttt{vX.X.X} for stable versions.

%%%%%%%%%%%%%%%%%%%%%%%%%%%%%%%%%%%%%%%%%%%%%%%%%%%%%%%%%%%%%%%%%%%%%%%%%%%%%%
%%%%%%%%%%%%%%%%%%%%%%%%%%%%%%%%%%%%%%%%%%%%%%%%%%%%%%%%%%%%%%%%%%%%%%%%%%%%%%
%%%%%%%%%%%%%%%%%%%%%%%%%%%%%%%%%%%%%%%%%%%%%%%%%%%%%%%%%%%%%%%%%%%%%%%%%%%%%%

\section{Related Projects}

Along with \ldbcfinbench, LDBC~\cite{DBLP:journals/sigmod/AnglesBLF0ENMKT14}
provides other benchmarks as well:

\begin{itemize}
      \item \ldbcsnb measures the performance of \emph{all systems relevant to
                  linked data} operating a social network.
      \item The Semantic Publishing Benchmark
            (SPB)~\cite{DBLP:conf/semweb/SpasicJP16} measures the performance of
            \emph{semantic databases} operating on RDF datasets.
      \item The Graphalytics
            benchmark~\cite{DBLP:journals/pvldb/IosupHNHPMCCSAT16} measures the
            performance of \emph{graph analysis} operations (\eg PageRank, local
            clustering coefficient).
\end{itemize}